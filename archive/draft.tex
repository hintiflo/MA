\documentclass{article}



\begin{document}

\subsubsection*{ Ziel: Board-Level-Simulation um }
\begin{itemize}
	\item SNR in analog-Pfaden zu verbessern
	\item Supply-rails entrauschen
	\item EMV abschätzen
	\item Schwingneigung detektieren
	\item Crosstalk abschätzen
\end{itemize}

\subsubsection*{Motivation} ein OPV-Board funktionierte mit einem TL071, zeigte mit OPA657 jedoch massive Schingneigung $\rightarrow$ die parasitäre Impedanz des Feedback-Pfads war das Problem und konnt durch vias innerhalb der Schleife gelöst werden

\subsubsection*{NonPlusUltra}
	Co-Simulation der Spice-Modelle mit den Feldgleichungen des PCBs, aus KiCAD heraus exportiert


\subsubsection*{mögliche Varianten}
\begin{enumerate}
\item
	modellierte Komponenten per ngspice cosimuliert mit PCB-Geometrie per EM-FieldSolver (FTDT in opemEMS/meep, oder FIT in mafia). stepwise mit Austausch der IO-Daten zw. spice/FieldSolver in jedem Simulationsschritt. Erfordert open-source solver um Datentausch per Simulationsschritt zu ermöglichen.
	
\item
	EM-Solver erzeugt aus der PCB-Geometrie alle Parasiten als R, L und C, diese gehen wieder zurück ins SPICE.
	
\item	IBIS-GUI $\leftrightarrow$ KiCAD für EMV
\end{enumerate}

\subsubsection*{Kandidaten}
\begin{itemize}
\item ngspice: \\
	+ opensource \\
	- kann keine encrypted models simulieren \\

\item LTSpice: \\
	+ nimmt encrypteds von LT und analogDev \\
	- nicht stepwise zu betreiben \\

\item openEMS, FTDT
\item meep, FTDT
\item MAFIA, FIT
\end{itemize}

\end{document}