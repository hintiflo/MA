\chapter{Abstract}

The company RECENDT GmbH develops and produces OCT-systems (optical coherence tomography), for which, as part of this master thesis, a part of the controlling system will be designed. For 2-dimensional measurement with OCT-systems, galvanometer-scanners are employed in an x/y-mode. These are highly-dynamical rotational drives for optical application with approximately 20° of angle for forward an back-rotation at rates up to 1kHz. Carrying mirrors, that are rotating along, these galvanometer-scanners can manipulate an optical path, typically from a focused coherent light-source, in 2 dimensions and therefore allow to scan whole surfaces with OCT-systems. \\

The chosen scanner-models require control signals to create scan-patterns, usually rectangles. Therefore, two synchronous ramp signals, a slow and a fast one are necessary. The task at hand is now, to program an existing microcontroller-hardware to form a two-channel arbitrary signal generator, called OCTane. This contains firmware modules to access digital-analogue-converters, trigger-units, for correct timing and the synchronisation. Furthermore USB-connectivity in a SCPI-style fashion has to be implemented, as the control of the unit shall be provided via USB. \\

Well established measures of quality for software development allow insightful analysis of the inspected software. These measures, for the bigger part, rely on an underlying operating system, and the inspected software to be compiled on the host-system or an equivalent one. To gain similar insight to cross-compiled bare-metal firmware, the application of mentioned quality-measures shall be researched. Obstacles are to expected from the necessary cross-compilation and the absence of an underlying operating-system.
