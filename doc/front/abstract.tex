\chapter{Abstract}

\TODO{vorläufige Fassung, vom Beginn der Masterarbeit}

Well established measures of quality for software development allow insightful analysis of the inspected software. These measures, for the bigger part, rely on an underlying operating system, and compilation of the inspected software on the host-system or an equivalent one. To gain similar insight to cross-compiled bare-metal firmware, the application of mentioned quality-measures shall be researched. Obstacles are expected from the necessary cross-compilation and the absence of an underlying operating-system. \\

This master thesis contains such a firmware project, that forms part of the controlling system of an OCT-systems (optical coherence tomography). For 2-dimensional measurement, OCT-systems require galvanometer-scanners, set up in an x/y-mode. These are highly-dynamical rotational drives for optical application with up to 20° of angle for forward and back-rotation at rates up to 1kHz. These galvanometer-scanners manipulate a light beam from a focused coherent light-source, by deflection via rotating mirrors. Manipulation of the light beam in two dimensions allows OCT-systems to scan areas, instead of single points. The RECENDT GmbH is an Austrian, non-university research institute specialized in non-destructive testing. This company is active in the development of OCT-Systems. \\

The chosen scanner-models require control signals to create scanning areas, which usually are of rectangular form. Therefore, two synchronous analogue ramp signals, a slow and a fast one, are necessary. The task of this thesis is, to program an existing microcontroller-hardware to form a two-channel arbitrary signal generator, called OCTane. This contains firmware modules to access digital-analogue-converters, trigger-units, for correct timing and synchronisation. Furthermore USB-connectivity in a SCPI-style is necessary, as the control of the unit happens via USB. Various types of code coverage determine the quality level of the resulting firmware. \\

