\chapter{Kurzfassung}

\begin{german}
Die Firma RECENDT GmbH entwickelt und baut OCT-Systeme (optical coherence tomography), für die im Rahmen dieser Masterarbeit ein Teil der Steuerung entworfen werden soll. Zur 2-dimensionalen Messung mit OCT-Systemen kommen Galvanometer-Scanner im X/Y-Betrieb zum Einsatz. Das sind hochdynamische Drehantriebe für optische Anwendungen, die mit einer Rate von rund 1kHz etwa 20° vor- und rückwärts rotieren können. Sie tragen mitrotierende Spiegel um den optischen Pfad in 2 Dimensionen auszulenken und somit flächige Scans zu ermöglichen. \\

Die ausgewählten Galvanometer-Scanner benötigen Steuersignale zur Erzeugung der Scan-Muster. Typischerweise sind dies zwei synchrone Rampen-Signale, eines schnell, eines langsam. Aufgabe ist es nun, auf bestehender Mikrocontroller-Hardware einen 2-kanaligen arbiträren Signalgenerator, namens OCTane zu programmieren. Dieser soll sowohl Rampen-Signale als auch arbiträr gewählte Signalformen erzeugen können. Dies beinhaltet FW-Module für die Digital-Analog-Wandler, Trigger-Einheit für das Timing sowie Synchronisation der Kanäle. Weiters ist die USB-Kommunikation per SCPI-Protokoll zu programmieren. Die Anbindung an eine übergeordnete Steuerung des OCT-Systems erfolgt per USB. \\

Etablierte Qualitätsmaße der Software-Entwicklung erlauben tiefe Einblicke in die untersuchte Software. Diese Maße stützen sich, großteils, auf ein unterliegendes Betriebssystem, sowie die Tatsache, daß die untersuchte Software auf dem Zielsystem oder einem Äquivalenten kompiliert wird. Um ähnlich tiefgreifenden Aufschluss über cross-kompilierte bare-metal Firmware zu erhalten, soll die Anwendbarkeit erwähnter Qualitätsmaße untersucht werden. Als hinderlich dabei ist zu erwarten, daß Firmware generell cross-kompiliert werden muss und eventuell kein unterstützendes Betriebssystem enthält.

% Die handelsüblichen Galvo-Scanner besitzen mechanische Trägheiten, die bei Scan-Raten ab 800Hz kein maximale Auslenkung mehr erlauben. Deshalb würde bei hohen Geschwindgkeiten der optische Messbereich eingeschränkt werden. Um auch bei höheren Scan-Raten volle Auslenkungen zu erreichen, soll versucht werden, mit adaptierten Steuersignalen die Trägheiten auszugleichen. \\
\end{german}